\documentclass{article}
    % General document formatting
    \usepackage[margin=0.7in]{geometry}
    \usepackage[parfill]{parskip}
    \usepackage[utf8]{inputenc}
    \usepackage{amsmath}
    \usepackage{amssymb}
    \usepackage{amsthm}
    \usepackage{tikz}
    \usepackage{fancyhdr}

\pagestyle{fancy}
\fancyhf{}
\rhead{Edgar Jacob Rivera Rios - A01184125}

\renewcommand{\labelenumi}{\alph{enumi})}
\renewcommand{\labelenumii}{\roman{enumii})}

\begin{document}
\section*{1.4.1 Proof by simple induction}
\begin{enumerate}
    \item Use simple induction to show that for every positive integer $n$, $5^n - 1$ is divisible by 4\\
    $5^{n}-1$ is divisible by 4\\
    Base case:
    \begin{align*}
        5^{1}-1 &= 4j\\
        &= 4 = 4j\\
        5^{k}-1 &= 4j\\
        5^{k}&= 4j+1
    \end{align*}
    Induction step:
    \begin{align*}
        &= 5^{k+ 1}-1\\
        &= 5^{k} \cdot 5 - 1\\
        &=(4j+1) \cdot 5 - 1\\
        &=(20j + 5) -1\\
        &=20j + 4\\
        &=4\cdot(5j + 1)\\
        &\implies true\ for\ n = k + 1
    \end{align*}
    \item Use simple induction to show that for every positive integer $n$, $n^3- n$ is divisible by 3. (Hint: In the induction step, you will need to make use of the arithmetic fact that $(k+1)^3 = k^3 + 3k^2 + 3k +1$)\\
    $n^3- n$ is divisible by 3\\
    Base case:
    \begin{align*}
        1^3- 1 &= 3j\\
        &= 0 = 3j\\
        k^{3}-k &= 3j\\
        k^{3}&= 3j+k
    \end{align*}
    Induction step:
    \begin{align*}
        &= (k + 1)^{3}-(k +1)\\
        &= (k^3 + 3k^2 + 3k +1) - (k +1)\\
        &= k^3 + 3k^2 + 2k\\
        &= (3j+k) + 3k^2 + 2k\\
        &= 3j+ 3k^2 + 3k\\
        &=3\cdot(j + k^2 + k)\\
        &\implies true\ for\ n = k + 1
    \end{align*}
    \item Show by simple induction that for every natural number $n$, $\sum_{i=0}^{n} 2^{i} = 2^{n+1} - 1$\\
    $f(n) = 2^{n+1} - 1 $\\
    Base case:
    \begin{align*}
        f(1) &= 2^{1+1} - 1\\
        f(1) &= 4 - 1\\
        f(1) &= 3\\
        f(k) &= 2^{k+1} - 1
    \end{align*}
    Induction step:
    \begin{align*}
        f(k +1) &= 2^{k + 2} - 1\\
        f(k +1) &= 2^1 + 2^2 ... + 2^{k} + 2^{k +1}\\
        &=f(k) + 2^{k +1}\\
        &=2^{k+1} - 1 + 2^{k +1}\\
        &=2(2^{k+1}) - 1\\
        &=2^{1} \cdot 2^{k+1} - 1\\
        &=2^{k+1 +1} - 1\\
        &=2^{k+2} - 1\\
        &\implies true\ for\ n = k + 1
    \end{align*}
\end{enumerate}
\section*{1.4.2: Definition by simple recursion}
\begin{enumerate}
    \item  Let $f: N \rightarrow N$ be the function defined by putting $f (0) = 0$ and $f(n+1) = n$ for all $n \in N$.
    \begin{enumerate}
        \item Evaluate this function bottom-up for all arguments 0–5.
        \begin{align*}
            f(0) &= 0\\
            f(1) &= 0\\
            f(2) &= 1\\
            f(3) &= 2\\
            f(4) &= 3\\
            f(5) &= 4
        \end{align*}
        \item Explain what f does by expressing it in explicit terms (i.e. without a recursion).\\
        This function is not recursive, it only returns the antecesor of the current value
    \end{enumerate}
    \item Let $f : N^{+} \rightarrow N$ be the function that takes each positive integer $n$ to the greatest natural number $p$ with $2^{p} \leq n$. Define this function by a simple recursion. (Hint: You will need to divide the recursion step into two cases.)
    \begin{align*}
        &f(n) = 0& &, when\ n = 1\\
        &f(n) = f(n-1)& &, when\ n > 1, \log_{2}(n) \notin N\\
        &f(n) = f(n-1) + 1& &, when\ n > 1, \log_{2}(n) \in N
    \end{align*}
    \item Let $g: NXN \rightarrow N$ be defined by putting $g(m,0)= m$ for all $m \in N$ and $g(m,n+1)=f (g(m,n))$ where $f$ is the function defined in part (a) of this exercise.
    \begin{enumerate}
        \item Evaluate $g(3,4)$ top-down.
        \begin{align*}
            g(3, 4) &= f(g(3, 3))\\
            &= f(f(g(3, 2)))\\
            &= f(f(f(g(3, 1))))\\
            &= f(f(f(f(g(3, 0)))))\\
            &= f(f(f(f(f(3)))))\\
            &= f(f(f(f(2))))\\
            &= f(f(f(1)))\\
            &= f(f(0))\\
            &= f(0)\\
            &= 0\\
        \end{align*}
        \item Explain what $g$ does by expressing it in explicit terms (i.e. without a recursion).\\
        It substracts the right item from the left, but if the remain is negative, it returns 0
    \end{enumerate}
\end{enumerate}

\section*{1.4.3: Proof by cumulative induction}
\begin{enumerate}
    \item Use cumulative induction to show that any postage cost of four or more pence can be covered by two-pence and five-pence stamps.\\
    $2x + 5y = n$, when $n \geq 4$\\
    Base case:
    \begin{align*}
        n &= 4 &4 &= 2(2) + 5(0)
    \end{align*}
    Induction step:
    \begin{align*}
        hypothesis &\rightarrow \forall j < k,\ j = 2x +5y\\
        goal &\rightarrow k = 2x + 3y\\
        case (1) &\rightarrow k\ is\ multiple\ of\ two \implies y = 0\ and\ x \in N\\
        case (2) &\rightarrow k\ is\ multiple\ of\ five \implies x = 0\ and\ y \in N\\
        case (3) &\rightarrow k\ is\ not\ multiple\ of\ any \implies y \in N\ and\ x \in N\\
    \end{align*}
    \item Use cumulative induction to show that for every natural number $n$, $F(n) \leq 2^{n}-1$, where $F$ is the Fibonacci function.\\
    $F(n) \leq 2^{n} -1$, \\
    Base case:
    \begin{align*}
        n &\leq 1 &1 &\leq 2^{1} - 1
    \end{align*}
    Induction step:
    \begin{align*}
        hypothesis &\rightarrow \forall j < k,\ F(j) \leq 2^{j} - 1\\
        goal &\rightarrow F(k) \leq 2^{k} -1\\
        case (1) &\rightarrow F(k)\ is\ greater\ than\ 2^{k} -1 \implies 2^{k+1} -1 < F(j +1)\\
        &Impossible\ given\ that\ F\ grows\ at\ F(k)\ and\ 2^{k} -1\ grows\ at\ double\ rate\\
        case (2) &\rightarrow F(k)\ is\ less\ or\ equal\ than\ 2^{k} -1e \implies This\ must\ be\ true\ then
    \end{align*}
    \item Calculate F(5) top-down, and then again bottom-up, where again $F$ is the Fibonacci function\\
    Top-down:
    \begin{align*}
        F(5) &= F(4) + F(3) \\
        F(5) &= (F(3) + F(2)) + (F(2) + F(1)) \\
        F(5) &= ((F(2) + F(1)) + (F(1) + F(0))) + ((F(1) + F(0)) + 1)\\
        F(5) &= (((F(1) + F(0)) + 1) + (1 + 0)) + ((1 + 0) + 1)\\
        F(5) &= (((1+ 0) + 1) + (1 + 0)) + ((1 + 0) + 1)\\
        F(5) &= ((1 + 1) + (1)) + (1 + 1)\\
        F(5) &= (2 + 1) + (2)\\
        F(5) &= 3 + 2\\
        F(5) &= 5
    \end{align*}
    Bottom-up
    \begin{align*}
        F(0) &= 0 \\
        F(1) &= 1 \\
        F(2) &= F(1) + F(0) = 1 + 0 = 1\\
        F(3) &= F(2) + F(1) = 1 + 1 = 2\\
        F(4) &= F(3) + F(2) = 2 + 1 = 3\\
        F(5) &= F(4) + F(3) = 3 + 2 = 5
    \end{align*}
    \item Express each of the numbers 14, 15 and 16 as a sum of $3s$ and/or $8s$. Using this fact in your basis, show by cumulative induction that every positive integer $n \leq 14$ may be expressed as a sum of $3s$ and/or $8s$.\\
    $n = 3x + 8y$, when $n > 14$\\
    Base case:
    \begin{align*}
        n &= 14 & 14 = 3(2) + 8(1)\\
        n &= 15 & 15 = 3(5) + 8(0)\\
        n &= 16 & 16 = 3(0) + 8(2)\\
    \end{align*}
    Induction step:
    \begin{align*}
        hypothesis &\rightarrow \forall j < k,\ j = 3x + 8y\\
        goal &\rightarrow k = 3x + 8y\\
        case (1) &\rightarrow k\ is\ multiple\ of\ three \implies y = 0\ and\ x \in N\\
        case (2) &\rightarrow k\ is\ multiple\ of\ eigth \implies x = 0\ and\ y \in N\\
        case (3) &\rightarrow k\ is\ not\ multiple\ of\ any \implies y \in N\ and\ x \in N\\
    \end{align*}
    \item Show by induction that for every natural number $n$, $A(1,n) = n+2$, where $A$ is the Ackermann function.
    \[   
    A(m,n) = 
    \begin{cases}
        n + 1 &\quad if\ m=0\\
        A(m-1, 1) &\quad if\ m >0\ and\ n=0\\
        A(m-1, A(m, n - 1)) &\quad if\ m >0\ and\ n >0\\
    \end{cases}
    \]
    Base case:
    \begin{align*}
        n &= 0 & A(1, 0) &= A(0, 1)\\
        & & &=2\\
        & & &=(0) +2\\
        n &= 1 & A(1, 1) &= A(0, A(1, 0))\\
        & & &=A(0, 2)\\
        & & &=3\\
        & & &=(1) +2\\
    \end{align*}
    Induction step:
    \begin{align*}
        hypothesis &\rightarrow \forall j < k,\ A(1,j) = j + 2\\
        goal &\rightarrow A(1,k) = k + 2\\
        A(1, k + 1) &= (k + 1) + 2\\
        &= A(0, A(1, k))\\
        &= A(1, k) +1\\
        &= (k + 2) +1\\
        A(1, k + 1) &= k + 3\\
        &\therefore A(1,k) = k + 2
    \end{align*}
\end{enumerate}

\end{document}