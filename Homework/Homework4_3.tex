\documentclass{article}
    % General document formatting
    \usepackage[margin=0.7in]{geometry}
    \usepackage[parfill]{parskip}
    \usepackage[utf8]{inputenc}
    \usepackage{mathrsfs}
    \usepackage{amsmath}
    \usepackage{amssymb}
    \usepackage{tikz}
    \usepackage{fancyhdr}
    \usepackage{multicol}

    \usetikzlibrary{positioning}

\pagestyle{fancy}
\fancyhf{}
\rhead{Edgar Jacob Rivera Rios - A01184125}

\renewcommand{\labelenumi}{\alph{enumi})}

\begin{document}
\section*{4.3.1}
Individuals filing federal income tax returns prior to March 31 received an average refund of \$1056. Consider the population of “last-minute” filers who mail their tax return during the last five days of the income tax period (typically April 10 to April 15).
\begin{enumerate}
  \item A researcher suggests that a reason individuals wait until the last five days is that on average these individuals receive lower refunds than do early filers. Develop appropriate hypotheses such that rejection of $H_{0}$ will support the researcher’s contention.
  \begin{align*}
    \text{Average of people that fill last days} &= \mu_{l}\\
    H_{0} &= \mu \leq \mu_{l}\\
    H_{\alpha} &= \mu > \mu_{l}\\
  \end{align*}

  \item For a sample of 400 individuals who filed a tax return between April 10 and 15, the sample mean refund was \$910. Based on prior experience a population standard deviation of $\sigma$ \$1600 may be assumed. What is the $p$-value?
  \begin{align*}
    \sigma &= 1600\\
    \mu &= 1056\\
    \bar{x} &= 910\\
    n &= 400\\
    \sigma_{\bar{x}} &= \frac{\sigma}{\sqrt{n}}\\
    \sigma_{\bar{x}} &= \frac{1600}{\sqrt{400}}\\
    \sigma_{\bar{x}} &= 80\\
    z &= \frac{\bar{x} - \mu}{\sigma_{\bar{x}}}\\
    z &= \frac{910 - 1056}{80}\\
    z &= -1.825\\
    p-value &= 0.0340005
  \end{align*}

  \item At $\alpha =.05$, what is your conclusion?

  As the p-value is less than $\alpha$, we can reject $H_{0}$, which means that $H_{\alpha}$ is true, which means that the individuals who mail the tax return in the last 5 days truly receive less than the ones who send them earlier

  \item Repeat the preceding hypothesis test using the critical value approach.

  \begin{align*}
    z &= -1.825\\
    \text{critical value} (z_{0}) &= -1.64485
  \end{align*}
  As $z < z_{0}$ we reject the null hypothesis

\end{enumerate}
\pagebreak

\section*{4.3.2}
In a study entitled How Undergraduate Students Use Credit Cards, it was reported that undergraduate students have a mean credit card balance of \$3173 (Sallie Mae, April 2009). This figure was an all-time high and had increased 44\% over the previous five years. Assume that a current study is being conducted to determine if it can be concluded that the mean credit card balance for undergraduate students has continued to increase compared to the April 2009 report. Based on previous studies, use a population standard deviation $\sigma =\$1000$.
\begin{enumerate}
  \item State the null and alternative hypotheses.
  \begin{align*}
    \text{Updated CC balance} &= \mu_{c}\\
    H_{0} &= \mu \geq \mu_{c}\\
    H_{\alpha} &= \mu < \mu_{c}\\
  \end{align*}

  \item What is the $p$-value for a sample of 180 undergraduate students with a sample mean credit card balance of \$3325?
  \begin{align*}
    \sigma &= 1000\\
    \mu &= 3173\\
    \bar{x} &= 3325\\
    n &= 180\\
    \sigma_{\bar{x}} &= \frac{\sigma}{\sqrt{n}}\\
    \sigma_{\bar{x}} &= \frac{1000}{\sqrt{180}}\\
    \sigma_{\bar{x}} &= 74.53559924999\\
    z &= \frac{\bar{x} - \mu}{\sigma_{\bar{x}}}\\
    z &= \frac{3325 - 3173}{74.53559924999}\\
    z &= 2.03929399547981\\
    p-value &= 0.020710347281277
  \end{align*}

  \item Using a .05 level of significance, what is your conclusion?

  As the $p$-value is smaller than .05, we can reject the null hypothesis, meaning that the credit card balance for undergraute students has continued rising

\end{enumerate}
\pagebreak

\section*{4.3.3}
Consider the following hypothesis test:
\begin{align*}
  H_{0}: \mu \leq 25\\
  H_{\alpha}: \mu > 25\\
\end{align*}
A sample of 40 provided a sample mean of 26.4. The population standard deviation is 6.
\begin{enumerate}
  \item Compute the value of the test statistic.

  \begin{align*}
    \sigma &= 6\\
    \mu &= 25\\
    \bar{x} &= 26.4\\
    n &= 40\\
    \sigma_{\bar{x}} &= \frac{\sigma}{\sqrt{n}}\\
    \sigma_{\bar{x}} &= \frac{6}{\sqrt{40}}\\
    \sigma_{\bar{x}} &= 0.9486832980505\\
    z &= \frac{\bar{x} - \mu}{\sigma_{\bar{x}}}\\
    z &= \frac{26.4 - 25}{0.9486832980505}\\
    z &= -1.47572957474524\\
  \end{align*}

  \item What is the $p$-value?
  \begin{align}
    p-value &= 0.070008251598585
  \end{align}

  \item At $\alpha = .01$, what is your conclusion?

  As the $p-value$ is greater than alpha,the null hypothesis cannot be rejected, so we don't have a definitive answer to this question

  \item What is the rejection rule using the critical value? What is your conclusion?

  The rejection rule is that $z < z_{0}$, but it this case $z =-1.47572957474524$ and $z_{0} = -2.32634787404084$, so the rejection does not hold

\end{enumerate}

\end{document}