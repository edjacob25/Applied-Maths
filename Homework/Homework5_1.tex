\documentclass{article}
    % General document formatting
    \usepackage[margin=0.7in]{geometry}
    \usepackage[parfill]{parskip}
    \usepackage[utf8]{inputenc}
    \usepackage{mathrsfs}
    \usepackage{amsmath}
    \usepackage{amssymb}
    \usepackage{tikz}
    \usepackage{fancyhdr}
    \usepackage{multicol}

    \usetikzlibrary{positioning}

\pagestyle{fancy}
\fancyhf{}
\rhead{Edgar Jacob Rivera Rios - A01184125}

\renewcommand{\labelenumi}{\alph{enumi})}

\begin{document}
\section*{5.1.1}
The Employment and Training Administration reported that the U.S. mean unemployment insurance benefit was \$238 per week (The World Almanac, 2003). A researcher in the state of Virginia anticipated that sample data would show evidence that the mean weekly unemployment insurance benefit in Virginia was below the national average.

\begin{enumerate}
  \item Develop appropriate hypotheses such that rejection of H0 will support the researcher’s contention.
  \item For a sample of 100 individuals, the sample mean weekly unemployment insurance benefit was \$231 with a sample standard deviation of \$80. What is the $p$-value?
  \item At $\alpha = .05$, what is your conclusion?
  \item Repeat the preceding hypothesis test using the critical value approach.
\end{enumerate}

\section*{5.1.2}
A shareholders’ group, in lodging a protest, claimed that the mean tenure for a chief executive office (CEO) was at least nine years. A survey of companies reported in The Wall Street Journal found a sample mean tenure of$\bar{X} = 7.27$ years for CEOs with a standard deviation of $s = 6.38$ years (The Wall Street Journal, January 2, 2007).
\begin{enumerate}
  \item Formulate hypotheses that can be used to challenge the validity of the claim made by the shareholders’ group.
  \item Assume 85 companies were included in the sample. What is the $p$-value for your hypothesis test?
  \item At $\alpha = .01$, what is your conclusion?

\end{enumerate}

\section*{5.1.3}
The Coca-Cola Company reported that the mean per capita annual sales of its beverages in the United States was 423 eight-ounce servings (Coca-Cola Company website, February 3, 2009). Suppose you are curious whether the consumption of Coca-Cola beverages is higher in Atlanta, Georgia, the location of Coca-Cola’s corporate headquarters.

A sample of 36 individuals from the Atlanta area showed a sample mean annual consumption of 460.4 eight-ounce servings with a standard deviation of $s = 101.9$ ounces.

Using $\alpha = .05$, do the sample results support the conclusion that mean annual consumption of Coca-Cola beverage products is higher in Atlanta?
\end{document}