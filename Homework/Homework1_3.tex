\documentclass{article}
    % General document formatting
    \usepackage[margin=0.7in]{geometry}
    \usepackage[parfill]{parskip}
    \usepackage[utf8]{inputenc}
    \usepackage{amsmath}
    \usepackage{amssymb}
    \usepackage{tikz}
    \usepackage{fancyhdr}

\pagestyle{fancy}
\fancyhf{}
\rhead{Edgar Jacob Rivera Rios - A01184125}

\renewcommand{\labelenumi}{\alph{enumi})}
\renewcommand{\labelenumii}{\roman{enumii})}

\begin{document}
\section*{1.3.1 Functions: Image, Closure}
\begin{enumerate}
    \item The floor function from $R_{+}$ into $N$ is defined by putting $\lfloor{x}\rfloor$ to be the largest integer less than or equal to $x$. What are the images under the floor function of the sets
    \begin{enumerate}
        \item $[0, 1] = \{x \in R: 0 \leq x \leq 1 \}$
        \item $[0, 1) = \{x \in R: 0 \leq x < 1 \}$
        \item $(0, 1] = \{x \in R: 0 < x \leq 1 \}$
        \item $(0, 1) = \{x \in R: 0 < x < 1 \}$
    \end{enumerate}
    \item Let $f : A\rightarrow A$ be a function from set $A$ into itself. Show that for all $X \subseteq A$, $f(X) \subseteq f[X]$, and give a simple example of the failure of the converse inclusion.
    \item Show that when then $f(A) \subseteq A$ then $f[A] = A$
    \item Show that for any partition of $A$, the function $f$ taking each element $a \in A$ to its cell is a function on $A$ into the power set $P(A)$ of $A$ with the partition as its range.
    \item Let $f: A \rightarrow B$ be a function from set $A$ into set $B$. Recall the ‘abstract inverse’ function$f^{-1}: B \rightarrow P(A)$ defined at the end of Slide 52 by putting $f^{-1}(b) = \{a \in A: f(a) = b\}$ for each $b \in B$.
    \begin{enumerate}
        \item Show that the collection of all sets for $b \in f(A) \subseteq B$ is a partition of $A$ in the sense defined in Chapter 2 of the David Makinson’s book.
        \item Is this still the case if we include in the collection the sets $f^{-1}(b)$ for $b \in B \setminus f(A)$?
    \end{enumerate}
\end{enumerate}
\section*{1.3.2 Injections, surjections, bijections }
\begin{enumerate}
    \item Is the floor function from $R_{+}$ into $N$ injective? (ii) Is it onto $N$?
    \item Show that the composition of two bijections is a bijection. You may make use of results of exercises in the previous slides on injectivity and surjectivity.
    \item Use the equinumerosity principle to show that there is never any bijection between a finite set and any of its proper subsets.
    \item Give an example to show that there can be a bijection between an infinite set and certain of its proper subsets.
    \item Use the principle of comparison to show that for finite sets $A$, $B$, if there are injective functions $f : A \rightarrow B$ and $g: B \rightarrow A$, then there is a bijection from $A$ to $B$. (Hint: Consider the superposition $g \circ f$ and establish that it provides for a desired bijection).
\end{enumerate}

\section*{1.3.3 Pigeonhole principle}
\begin{enumerate}
    \item If a set $A$ is partitioned into n cells, how many distinct elements of $A$ need to be selected to guarantee that at least two of them are in the same cell?
    \item Let $K =\{1,2,3,4,5,6,7,8\}$. How many distinct numbers must be selected from $K$ to guarantee that there are two of them that sum to 9? (Hint: Let $A$ be the set of all unordered pairs $(x,y)$ with $x,y \in K$ and $x+y = 9$. Check that this set forms a partition of K and apply the preceding part of the exercise).
\end{enumerate}

\section*{1.3.4 Handy functions}
\begin{enumerate}
    \item Let $f: A \rightarrow B$ and $g: B \rightarrow C$.
    \begin{enumerate}
        \item Show that if at least one of $f$, $g$ is a constant function, then $g \circ f : A \rightarrow C$ is a constant function.
        \item If $g \circ f : A \rightarrow C$ is a constant function, does it follow that at least one of $f$,$g$ is a constant function (give a verification or a counterexample).
    \end{enumerate}
\end{enumerate}
\end{document}