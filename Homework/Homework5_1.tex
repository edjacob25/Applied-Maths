\documentclass{article}
    % General document formatting
    \usepackage[margin=0.7in]{geometry}
    \usepackage[parfill]{parskip}
    \usepackage[utf8]{inputenc}
    \usepackage{mathrsfs}
    \usepackage{amsmath}
    \usepackage{amssymb}
    \usepackage{tikz}
    \usepackage{fancyhdr}
    \usepackage{multicol}

    \usetikzlibrary{positioning}

\pagestyle{fancy}
\fancyhf{}
\rhead{Edgar Jacob Rivera Rios - A01184125}

\renewcommand{\labelenumi}{\alph{enumi})}

\begin{document}
\section*{5.1.1}
The Employment and Training Administration reported that the U.S. mean unemployment insurance benefit was \$238 per week (The World Almanac, 2003). A researcher in the state of Virginia anticipated that sample data would show evidence that the mean weekly unemployment insurance benefit in Virginia was below the national average.

\begin{enumerate}
  \item Develop appropriate hypotheses such that rejection of H0 will support the researcher’s contention.
  \begin{align*}
    H_{0} &= \mu \geq \mu_{l}\\
    H_{\alpha} &= \mu < \mu_{l}\\
  \end{align*}

  \item For a sample of 100 individuals, the sample mean weekly unemployment insurance benefit was \$231 with a sample standard deviation of \$80. What is the $p$-value?
  \begin{align*}
    s &= 80\\
    \mu &= 238\\
    \bar{x} &= 231\\
    n &= 100\\
    s_{\bar{x}} &= \frac{s}{\sqrt{n}}\\
    s_{\bar{x}} &= \frac{80}{\sqrt{100}}\\
    s_{\bar{x}} &= 8\\
    t &= \frac{\bar{x} - \mu}{s_{\bar{x}}}\\
    t &= \frac{231 - 238}{8}\\
    t &= -0.875\\
    p-value &= 0.19184589
  \end{align*}

  \item At $\alpha = .05$, what is your conclusion?

  As the $p-value$ is greater than 0.05, we cannot reject $H_{0}$

  \item Repeat the preceding hypothesis test using the critical value approach.

  \begin{align*}
    t &= -0.875\\
    \text{critical value} (t_{0}) &= -1.66039115
  \end{align*}
  As $t < t_{0}$ does not hold, we cannot reject $h_{0}$

\end{enumerate}
\pagebreak

\section*{5.1.2}
A shareholders’ group, in lodging a protest, claimed that the mean tenure for a chief executive office (CEO) was at least nine years. A survey of companies reported in The Wall Street Journal found a sample mean tenure of$\bar{X} = 7.27$ years for CEOs with a standard deviation of $s = 6.38$ years (The Wall Street Journal, January 2, 2007).
\begin{enumerate}
  \item Formulate hypotheses that can be used to challenge the validity of the claim made by the shareholders’ group.
  \begin{align*}
    H_{0} &= \mu \geq 9\\
    H_{\alpha} &= \mu < 9 \\
  \end{align*}

  \item Assume 85 companies were included in the sample. What is the $p$-value for your hypothesis test?
  \begin{align*}
    s &= 6.38\\
    \mu_{0} &= 9\\
    \bar{x} &= 7.27\\
    n &= 85\\
    s_{\bar{x}} &= \frac{s}{\sqrt{n}}\\
    s_{\bar{x}} &= \frac{6.38}{\sqrt{85}}\\
    s_{\bar{x}} &= 0.692008160441513\\
    t &= \frac{\bar{x} - \mu_{0}}{s_{\bar{x}}}\\
    t &= \frac{7.27 - 9}{0.692008160441513}\\
    t &= -2.49997051898381\\
    p-value &= 0.007182545042323
  \end{align*}

  \item At $\alpha = .01$, what is your conclusion?

  As the $p-value$ is less than 0.01, we can reject $H_{0}$, which means that the CEO's mean tenure is less than 9 years

\end{enumerate}
\pagebreak

\section*{5.1.3}
The Coca-Cola Company reported that the mean per capita annual sales of its beverages in the United States was 423 eight-ounce servings (Coca-Cola Company website, February 3, 2009). Suppose you are curious whether the consumption of Coca-Cola beverages is higher in Atlanta, Georgia, the location of Coca-Cola’s corporate headquarters.

A sample of 36 individuals from the Atlanta area showed a sample mean annual consumption of 460.4 eight-ounce servings with a standard deviation of $s = 101.9$ ounces.

Using $\alpha = .05$, do the sample results support the conclusion that mean annual consumption of Coca-Cola beverage products is higher in Atlanta?

\begin{align*}
  H_{0} &= \mu \leq 423\\
  H_{\alpha} &= \mu > 423 \\
  s &= 101.9\\
  \mu_{0} &= 423\\
  \bar{x} &= 460.4\\
  n &= 36\\
  s_{\bar{x}} &= \frac{s}{\sqrt{n}}\\
  s_{\bar{x}} &= \frac{101.9}{\sqrt{36}}\\
  s_{\bar{x}} &= 16.9833333333333\\
  t &= \frac{\bar{x} - \mu_{0}}{s_{\bar{x}}}\\
  t &= \frac{460.4 - 423}{16.9833333333333}\\
  t &= 2.20215897939156\\
  p-value &= 0.01716737427853
\end{align*}

As the $p-value$ is less than 0.05, we can reject $H_{0}$, which means that the Coke consumption it's higher in Atlanta than in the rest of the USA

\end{document}