\documentclass{article}
    % General document formatting
    \usepackage[margin=0.7in]{geometry}
    \usepackage[parfill]{parskip}
    \usepackage[utf8]{inputenc}
    \usepackage{mathrsfs}
    \usepackage{amsmath}
    \usepackage{amssymb}
    \usepackage{tikz}
    \usepackage{fancyhdr}
    \usepackage{multicol}

    \usetikzlibrary{positioning}

\pagestyle{fancy}
\fancyhf{}
\rhead{Edgar Jacob Rivera Rios - A01184125}

\renewcommand{\labelenumi}{\alph{enumi})}

\begin{document}
\section*{3.1.1}
A survey question for a sample of 150 individuals yielded 75 \textbf{Yes} responses, 55 \textbf{No} responses, and 20 \textbf{No Opinions}.
\begin{enumerate}
  \item What is the point estimate of the proportion in the population who respond \textbf{Yes}?
  \begin{equation*}
    \bar{p} = \frac{x}{n} = \frac{75}{150} = 0.5
  \end{equation*}
  \item What is the point estimate of the proportion in the population who respond \textbf{No}?
  \begin{equation*}
    \bar{p} = \frac{x}{n} = \frac{55}{150} = \frac{11}{30} = 0.36666666
  \end{equation*}
\end{enumerate}

\section*{3.1.2}
Many drugs used to treat cancer are expensive. BusinessWeek reported on the cost per treatment of Herceptin, a drug used to treat breast cancer(BusinessWeek, January 30, 2006).

Typical treatment costs (in dollars) for Herceptin are provided by a simple random sample of 10 patients.
\begin{center}
  \textbf{\$4,376 \$5,578 \$2,717 \$4,920 \$4,495 \$4,798 \$6,446 \$4,119 \$4,237 \$3,814}
\end{center}

\begin{enumerate}
  \item Develop a point estimate of the mean cost per treatment with Herceptin
  \begin{equation*}
    \bar{x} = \frac{1}{n} \sum_{i=1}^{n} x_{i} = \frac{45500}{10} = \$4,550.00
  \end{equation*}
  \item Develop a point estimate of the standard deviation of the cost per treatment with Herceptin.
  \begin{equation*}
    s = \sqrt{\frac{1}{n - 1} \sum_{i=1}^{n}(x_{i} - \bar{x})^{2}} = \sqrt{\frac{9,068,620.00}{9}}= \$1003.80498327337
  \end{equation*}
\end{enumerate}

\section*{3.1.3}
The American Association of Individual Investors (AAII) polls its subscribers on a weekly basis to determine the number who are bullish, bearish, or neutral on the short-term prospects for the stock market. Their findings for the week ending March 2, 2006, are consistent with the following sample results (AAII website, March 7, 2006).
\begin{table}[h!]
  \centering
  \begin{tabular}{c c c}
    Bullish 409 &Neutral 299 &Bearish 291\\
  \end{tabular}
\end{table}
Develop a point estimate of the following population parameters.
\begin{enumerate}
  \item The proportion of all AAII subscribers who are bullish on the stock market.
  \begin{equation*}
    \bar{p} = \frac{x}{n} = \frac{409}{999} = 0.409
  \end{equation*}
  \item The proportion of all AAII subscribers who are neutral on the stock market.
  \begin{equation*}
    \bar{p} = \frac{x}{n} = \frac{299}{999} = 0.299
  \end{equation*}
  \item The proportion of all AAII subscribers who are bearish on the stock market.
  \begin{equation*}
    \bar{p} = \frac{x}{n} = \frac{291}{999} = 0.291
  \end{equation*}
\end{enumerate}

\section*{3.1.4}
A simple random sample of 5 months of sales data provided the following information:
\begin{table}[h!]
  \centering
  \begin{tabular}{c c c c c c}
    Month: &1 &2 &3 &4 &5 \\
    Units Sold: &94 &100 &85 &94 &92\\
  \end{tabular}
\end{table}
\begin{enumerate}
  \item Develop a point estimate of the population mean number of units sold per month.
  \begin{equation*}
    \bar{x} = \frac{1}{n} \sum_{i=1}^{n} x_{i} = \frac{465}{5} = 93
  \end{equation*}
  \item Develop a point estimate of the population standard deviation.
  \begin{equation*}
    s = \sqrt{\frac{1}{n - 1} \sum_{i=1}^{n}(x_{i} - \bar{x})^{2}} = \sqrt{\frac{116}{4}}= 5.3851648071345
  \end{equation*}
\end{enumerate}
\end{document}